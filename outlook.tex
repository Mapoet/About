\begin{rubric}{展望}

\entry*[]基于行云及鸿雁计划实施,搭载于低轨卫星上的多样化载荷通过多源多途径方式更好地服务于军民生产计划与实施,将是未来热门方向。以低轨卫星与导航星联合服务为例,低轨卫星通过定位天线与掩星天线可以全天候全球性捕获不同高度范围的空间环境,获得大气及电离层物理参数进行空间环境预警。电离层观测可以为电离层乃至等离子层监测及物理提供数据支撑;全尺度中性大气观测可以为数值天气及气象服务提供评估及预测保障,更好地服务于气象监测、宇航保障、及渔/船/港口业务定制化服务。而这些产品也可基于星基分发电离层及对流层改正量等支持星基增强服务,可应用于无人机、无人船、机器人及自动驾驶等方面服务。本人在本科学习了完备的大地测量及导航知识,在硕博阶段主要基致力于大气及电离层的地基观测与星载观测(掩星观测)数据处理及融合,未来拟继续完善电离层及大气的多源观测数据探测及处理,经验模型构建及数值同化方面的理论与应用研究。

%http://renshi.nwpu.edu.cn/ningbozhaopin0825.pdf
%https://ningbo.nwpu.edu.cn/info/1009/1387.htm
\end{rubric}