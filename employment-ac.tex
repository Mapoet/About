\begin{rubric}{工作经历}
\entry*[2021-2 至今]\textbf{博士后}\space{天津大学,天津}.

研究内容:GNSS-R/-RO技术设备研制及数据应用

研究方向: 1. 论证卫星轨道建设卫星分布,覆盖范围及对应的电离层及大气掩星分布与时空分辨率; 2. 一体化GNSS-RO载荷研制及掩星数据反演; 3. GNSS-R载荷研制及数据反演; 4. 大气及电离层掩星数据与多源数据融合。

\entry*[2020 至今]\textbf{算法研究员,兼职}\space{天津云遥宇航科技公司,天津}.

1.主要进行大气及电离层掩星数据论证及模拟(卫星轨道设计及其时空分辨率分析); 2.与硬件研发合作验证接收机性能与指标(钟差,信噪比,信号连续性等); 3.实现掩星开环跟踪及反掩算法,并进行山基及信号源数据测试。

\entry*[2016-09 至 2020年7月]\space\textbf{完成人,博士研究生}\space{中国科学院上海天文台}.

电离层同化算法误差分析,及多源数据在电离层建模中的使用,也是本人在博士阶段主要的研究内容。在2016年9月,开始自己的博士阶段研究,主要研究方向是电离层多源数据处理及电离层建模。通过研读文献,在获得电离层地基,空基及掩星数据之后,依照数据特征进行综合建模与利用,获得高精度电离层4D产品。在研究期间,重点研究了电离层同化中的误差源、同化方式,完成了6-7篇相关论文(其中4篇一作,两篇SCI),并完成了博士论文。

参与的项目:
\begin{itemize}
      \item 国家自然科学基金联合项目,基于GNSS/LEO电离层掩星观测资料改进电离层和等离子模型研究
      \item 上海市科学技术委员会计划项目,GNSS掩星数据处理和接收机原理样机研制
      \item 503项目,LEO-LEO反演,大气全谱反演技术开发
      \item 中国地震局地壳研究所项目,垂直TEC地面验证方法研究及其软件编制
      \item 航天东方红卫星有限公司项目,三频信标系统有效载荷任务分析软件
      \item 中国地震局地震研究所项目,基于卡尔曼滤波技术构建电离层三维结构
\end{itemize}

\entry*[2015-06 至 2016-09]\space\textbf{主要完成人员,博士研究生}\space{中国科学院上海天文台}.

该项目承接于东方红公司。针对卫星信标电离层探测载荷任务分析与指标分解的任务,模拟发生信标电离层探测弧段的情况、模拟无线电信号在卫星-地面站之间电离层的传播过程和传播误差;基于上述仿真数据提取卫星信标电离层的相对TEC;模拟分析不同观测几何构型、输入模型条件、网格划分、台站布设情况下,对卫星路径以下的电离层剖面层析探测;对模拟数据与模型数据进行分析和比较,为三频信标探测数据处理的业务化提供一个比较成熟的平台。在该项目中,主要负责核心模块(以fortran语言实现)调试,接口一致性修改实现软件界面开发及批处理功能(shell,C/C++),以及软件交付材料准备及软件验收。

\entry*[2013-06 至 2014-08]\space\textbf{核心功能研发人员,学生}\space{中南大学}.

项目名称"惠州现代测绘基准服务平台“,该项目承接于惠州市国土局,希望达建立坐标基准转换综合平台,实现各部门各县区基于转换工具的坐标成果、高程成果的转换;需要实现城市数据框架平台内,基于坐标转换插件的各种数据格式的图件及数据成果的相互转换。在此次项目中,我完成了以下任务: ● 通过C++及COM技术,完成了支持多种符合测绘精度标准的多种坐标系转化算法,实现了对基于C\#开发的C/S平台与JS开发的B/S平台的支持; ● 通过SQL Server,设计了项目中使用的各种坐标数据及转化参数的数据库,并集成到了C/S及B/S平台 ● 参与并配合软件界面设计,完成各个模块的功能测试。

\entry*[2014-03 至 2014-06]\space\textbf{算法实现者,学生}\space{中南大学}

基于matlab自主研究SVM及ANN:1.基于词袋数思路,通过对标准图像库图片进行Sift特征提取与描述,然后使用聚类算法生成视觉词汇表,最后训练分类器(SVM)并进行分类检测。2.通过对ORL标准人脸库使用SVD进行特征提取,并通过神经网络进行训练与人脸识别;在研究中分析了对灰度图形是否进行傅利叶变换以及神经网络的参数(层数,神经元数目,核函数等)对人脸识别准确性的影响。具体项目见:https://github.com/Mapoet/Sift-SVM.git及https://github.com/Mapoet/SVD-ANN.git。

\end{rubric}


